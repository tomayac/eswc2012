\documentclass{llncs}

%%%%%%%%%% preamble %%%%%%%%%%

% margins
\usepackage[left=1.8cm, right=1.8cm, top=2.54cm, bottom=2.54cm]{geometry}

\usepackage[utf8]{inputenc}

% autoref command
\usepackage[pdftex,urlcolor=black,colorlinks=true,linkcolor=black,citecolor=black]{hyperref}
\def\sectionautorefname{Section}
\def\subsectionautorefname{Subsection}
\def\figureautorefname{Fig.}
\def\subfigureautorefname{Fig.}

% listings and Verbatim environment
\usepackage{fancyvrb}
\usepackage{relsize}
\RecustomVerbatimCommand{\Verb}{Verb}{fontsize=\smaller}
\RecustomVerbatimEnvironment{Verbatim}{Verbatim}{fontsize=\fontsize{7.5pt}{9pt}}
\usepackage{listings}
\lstloadlanguages{SPARQL} 
\lstdefinelanguage{RDF}{morekeywords=@prefix}
\lstset{frame=bottomline,captionpos=b,numberbychapter=false,
        aboveskip=0em,belowskip=.5em,abovecaptionskip=.3em,belowcaptionskip=.5em,
        basicstyle=\ttfamily\fontsize{7pt}{8.5pt}\selectfont,stringstyle=\em,showstringspaces=false}

% for multiline comments
\usepackage{verbatim} 

% abstract keywords
\newcommand{\keywords}[1]{\par\addvspace\baselineskip
\noindent\keywordname\enspace\ignorespaces#1}

% todo macro
\usepackage{color}
\newcommand{\todo}[1]{\noindent\textcolor{red}{{\bf \{TODO} #1{\bf \}}}}

% references
\usepackage{cite}

% inclusion of graphics and PDF documents
\usepackage[pdftex]{graphicx}
\usepackage[final]{pdfpages}
\usepackage[bf,small]{subfigure}
\setlength{\floatsep}{1em}
\setlength{\dblfloatsep}{1em}
\setlength{\textfloatsep}{1em}
\setlength{\dbltextfloatsep}{1em}
\setlength{\belowcaptionskip}{-1em}

% better typography
\usepackage[tracking, spacing, kerning]{microtype}

% drawings
\usepackage{tikz}
\usepackage{tkz-graph}
\usetikzlibrary{matrix,arrows,decorations.pathmorphing,shapes}

\newcommand{\bulletnumber}[1]{%
   \tikz[baseline=-2.5]%
   \node[circle,text=white,fill=gray,anchor=west,inner sep=1pt] {\rm #1};%
}

% algorithm typesetting
\usepackage[small,bf]{caption}
\usepackage{algorithmic}
\usepackage{float}
\newfloat{algorithm}{tpbh}{}
\floatname{algorithm}{Algorithm}
\def\algorithmautorefname{Algorithm}
\newcommand{\algorithmtop}{\rule{\columnwidth}{.4pt}\vspace{-1.3em}}
\newcommand{\algorithmbottom}{\vspace{-0.6em}\rule{\columnwidth}{.4pt}\vspace{-0.6em}}

% equations typesetting
\usepackage{amsmath}
% add parenthesis around equation references
\makeatletter
\def\tagform@#1{\maketag@@@{\ignorespaces#1\unskip\@@italiccorr}}
\let\orgtheequation\theequation
\def\theequation{(\orgtheequation)}
\makeatother

% lay-out
\usepackage{needspace}

% make sure that OWL-S does not get hyphenated
\newcommand{\owls}{\mbox{OWL-S}}

%%%%%%%%%% document %%%%%%%%%%

\begin{document}
% For every picture that defines or uses external nodes, you'll have to
% apply the 'remember picture' style. To avoid some typing, we'll apply
% the style to all pictures.
\tikzstyle{every picture}+=[remember picture]

\mainmatter


%%%%%%%%%% title %%%%%%%%%%

\title{Confomaton: A Conference Enhancer for the Aggregation and Reconciliation of Data from Media Clouds}

\author{Houda Khrouf\inst{1}, Ghislain Atemezing\inst{1}, Guiseppe Rizzo\inst{1}, Thomas Steiner\inst{2} and Raphaël Troncy\inst{1}}

\institute{EURECOM, Sophia Antipolis, France, \email{\{houda.khrouf, auguste.atemezing, guiseppe.rizzo, raphael.troncy\}@eurecom.fr}
 \and Universitat Polit\`{e}cnica de Catalunya, Department LSI, 08034 Barcelona, Spain, \email{tsteiner@lsi.upc.edu}
}

\maketitle

% use Courier from this point onward (used CM for author e-mails), and smaller in-text URIs
\renewcommand{\ttdefault}{pcr}
\renewcommand\UrlFont{\smaller\tt}


%%%%%%%%%% abstract %%%%%%%%%%

\begin{abstract}
  A common practice in the Semantic Web community is to publish information in a structured way.
  For conferences, such information can include papers that were presented, people who attended, and other things that have to do with the main conferences and workshops.
  Typically a format such as RDF, ready to be consumed by any Web application, is used, as is the case with the Semantic Web Dog Food Corpus.
  However, conferences are venues where participants spread digital information such as tweets, pictures, slides, and share geo-tagged information on the Web.
  An important challenge is to link user generated content from public services such as Twitter, TwitPic, Flickr, Slideshare to structured data provided by the Semantic Web Dog Food Corpus.
  The general idea is to use the approach to identify, aggregate and reconciliate data for attaching media (photos, videos, slides) and microposts to part of structured data.
  To address the above challenge, we developed an application, deployed on the top of Semantic Web technologies and conducted a real test case during the ISWC'11 conference.
  To assess the validity of our approach, we provide real case statistics gathered from the users.
\end{abstract}

%%%%%%%%%% body %%%%%%%%%%

\section{Introduction} \label{sec:introduction}
Just for testing \cite{confomaton}.
- introduction
* general introduction on the Web of Data
* linked data
* how media service publish information on the Web
* though aggregation an reconciliation of data spread in the Web

- background
* list different technologies and explain them in different subsections

- confomaton
* use case: ISWC'11
* screenshots
* some statistics about participants, tweets, pictures, ..., just to highlight the importance to aggregate those data. If not linked, data could disappear or at least have less value

\section{Related Work} \label{sec:relatedwork}
* for each methodolody, describe what it the relative state of the art

\section{Discussion and Conclusion} \label{sec:discussionconclusion}
* detail the lesson learned during the ISWC'11
* what are future plans


\section*{Acknowledgements} \label{sec:acknowledgements}

% back to normal size Computer Modern for URLs in bibliography
\renewcommand{\ttdefault}{cmvtt}
\renewcommand\UrlFont\tt

\bibliographystyle{abbrv}
\bibliography{eswc2012}

\end{document}  